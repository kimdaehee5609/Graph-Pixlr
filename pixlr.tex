%	-------------------------------------------------------------------------------
% 
% 
% 
%	\chapter{메뉴의 구성} 
% 
% 
% 
% 
% 
% 
%	-------------------------------------------------------------------------------
	\documentclass[12pt, a4paper, oneside]{book}
%	\documentclass[12pt, a4paper, landscape, oneside]{book}

		% --------------------------------- 페이지 스타일 지정
		\usepackage{geometry}
%		\geometry{landscape=true	}
		\geometry{top 		=10em}
		\geometry{bottom	=10em}
		\geometry{left		=5em}
		\geometry{right		=5em}
%		\geometry{left		=4em}
%		\geometry{right		=4em}

		\geometry{headheight	=4em} % 머리말 설치 높이
		\geometry{headsep		=2em} % 머리말의 본문과의 띠우기 크기
		\geometry{footskip		=4em} % 꼬리말의 본문과의 띠우기 크기
% 		\geometry{showframe}
	
%		paperwidth 	= left + width + right (1)
%		paperheight 	= top + height + bottom (2)
%		width 		= textwidth (+ marginparsep + marginparwidth) (3)
%		height 		= textheight (+ headheight + headsep + footskip) (4)



		%	===================================================================
		%	package
		%	===================================================================
%			\usepackage[hangul]{kotex}				% 한글 사용
			\usepackage{kotex}					% 한글 사용
			\usepackage[unicode]{hyperref}			% 한글 하이퍼링크 사용

		% ------------------------------ 수학 수식
			\usepackage{amssymb,amsfonts,amsmath}	% 수학 수식 사용
			\usepackage{mathtools}				% amsmath 확장판

			\usepackage{scrextend}				% 
		

		% ------------------------------ LIST
			\usepackage{enumerate}			%
			\usepackage{enumitem}			%
			\usepackage{tablists}				%	수학문제의 보기 등을 표현하는데 사용
										%	tabenum


		% ------------------------------ table 
			\usepackage{longtable}			%
			\usepackage{tabularx}			%
			\usepackage{tabu}				%




		% ------------------------------ 
			\usepackage{setspace}			%
			\usepackage{booktabs}		% table
			\usepackage{color}			%
			\usepackage{multirow}			%
			\usepackage{boxedminipage}	% 미니 페이지
			\usepackage[pdftex]{graphicx}	% 그림 사용
			\usepackage[final]{pdfpages}		% pdf 사용
			\usepackage{framed}			% pdf 사용

			
			\usepackage{fix-cm}	
			\usepackage[english]{babel}
	
		%	=======================================================================================
		% 	tikz package
		% 	
		% 	--------------------------------- 	
			\usepackage{tikz}%
			\usetikzlibrary{arrows,positioning,shapes}
			\usetikzlibrary{mindmap}			
			

		% --------------------------------- 	page
			\usepackage{afterpage}		% 다음페이지가 나온면 어떻게 하라는 명령 정의 패키지
%			\usepackage{fullpage}			% 잘못 사용하면 다 흐트러짐 주의해서 사용
%			\usepackage{pdflscape}		% 
			\usepackage{lscape}			%	 


			\usepackage{blindtext}
	
		% --------------------------------- font 사용
			\usepackage{pifont}				%
			\usepackage{textcomp}
			\usepackage{gensymb}
			\usepackage{marvosym}



		% Package --------------------------------- 

			\usepackage{tablists}				%


		% Package --------------------------------- 
			\usepackage[framemethod=TikZ]{mdframed}				% md framed package
			\usepackage{smartdiagram}								% smart diagram package



		% Package ---------------------------------    연습문제 

			\usepackage{exsheets}				%

			\SetupExSheets{solution/print=true}
			\SetupExSheets{question/type=exam}
			\SetupExSheets[points]{name=point,name-plural=points}


		% --------------------------------- 페이지 스타일 지정

		\usepackage[Sonny]		{fncychap}

			\makeatletter
			\ChNameVar	{\Large\bf}
			\ChNumVar	{\Huge\bf}
			\ChTitleVar		{\Large\bf}
			\ChRuleWidth	{0.5pt}
			\makeatother

%		\usepackage[Lenny]		{fncychap}
%		\usepackage[Glenn]		{fncychap}
%		\usepackage[Conny]		{fncychap}
%		\usepackage[Rejne]		{fncychap}
%		\usepackage[Bjarne]	{fncychap}
%		\usepackage[Bjornstrup]{fncychap}

		\usepackage{fancyhdr}
		\pagestyle{fancy}
		\fancyhead{} % clear all fields
		\fancyhead[LO]{\footnotesize \leftmark}
		\fancyhead[RE]{\footnotesize \leftmark}
		\fancyfoot{} % clear all fields
		\fancyfoot[LE,RO]{\large \thepage}
		%\fancyfoot[CO,CE]{\empty}
		\renewcommand{\headrulewidth}{1.0pt}
		\renewcommand{\footrulewidth}{0.4pt}
	
	
	
		%	--------------------------------------------------------------------------------------- 
		% 	tritlesec package
		% 	
		% 	
		% 	------------------------------------------------------------------ section 스타일 지정
	
			\usepackage{titlesec}
		
		% 	----------------------------------------------------------------- section 글자 모양 설정
			\titleformat*{\section}					{\large\bfseries}
			\titleformat*{\subsection}				{\normalsize\bfseries}
			\titleformat*{\subsubsection}			{\normalsize\bfseries}
			\titleformat*{\paragraph}				{\normalsize\bfseries}
			\titleformat*{\subparagraph}				{\normalsize\bfseries}
	
		% 	----------------------------------------------------------------- section 번호 설정
			\renewcommand{\thepart}				{\arabic{part}.}
			\renewcommand{\thesection}				{\arabic{section}.}
			\renewcommand{\thesubsection}			{\thesection\arabic{subsection}.}
			\renewcommand{\thesubsubsection}		{\thesubsection\arabic{subsubsection}}
			\renewcommand\theparagraph 			{$\blacksquare$ \hspace{3pt}}

		% 	----------------------------------------------------------------- section 페이지 나누기 설정
			\let\stdsection\section
			\renewcommand\section{\newpage\stdsection}



		%	--------------------------------------------------------------------------------------- 
		% 	\titlespacing*{commandi} {left} {before-sep} {after-sep} [right-sep]		
		% 	left
		%	before-sep		:  수직 전 간격
		% 	after-sep	 	:  수직으로 후 간격
		%	right-sep

			\titlespacing*{\section} 			{0pt}{1.0em}{1.0em}
			\titlespacing*{\subsection}	  		{0ex}{1.0em}{1.0em}
			\titlespacing*{\subsubsection}		{0ex}{1.0em}{1.0em}
			\titlespacing*{\paragraph}			{0em}{1.5em}{1.0em}
			\titlespacing*{\subparagraph}		{4em}{1.0em}{1.0em}
	
		%	\titlespacing*{\section} 			{0pt}{0.0\baselineskip}{0.0\baselineskip}
		%	\titlespacing*{\subsection}	  		{0ex}{0.0\baselineskip}{0.0\baselineskip}
		%	\titlespacing*{\subsubsection}		{6ex}{0.0\baselineskip}{0.0\baselineskip}
		%	\titlespacing*{\paragraph}			{6pt}{0.0\baselineskip}{0.0\baselineskip}
	

		% --------------------------------- recommend		섹션별 페이지 상단 여백
		\newcommand{\SectionMargin}				{\newpage  \null \vskip 2cm}
		\newcommand{\SubSectionMargin}			{\newpage  \null \vskip 2cm}
		\newcommand{\SubSubSectionMargin}		{\newpage  \null \vskip 2cm}


		%	--------------------------------------------------------------------------------------- 
		% 	toc 설정  - table of contents
		%	--------------------------------------------------------------------------------------- 
		%	--------------------------------------------------------------------------------------- 
		%	--------------------------------------------------------------------------------------- 
		%	--------------------------------------------------------------------------------------- 
		%	--------------------------------------------------------------------------------------- 
		%	--------------------------------------------------------------------------------------- 
		%	--------------------------------------------------------------------------------------- 
		% 	
		% 	
		% 	----------------------------------------------------------------  문서 기본 사항 설정 문단번호깊이
			\setcounter{secnumdepth}{5} 		% 문단 번호 깊이
			\setcounter{tocdepth}{0} 			% 문단 번호 깊이 - 목차 출력시 출력 범위
%			\setcounter{tocdepth}{2} 			% 문단 번호 깊이 - 목차 출력시 출력 범위
%			\setcounter{tocdepth}{-1} 			% 문단 번호 깊이 - 목차 출력시 출력 범위
			%	-1 part
			%	0	chapter
			%	1	section
			%	2	subsection
			%	3	subsubsection
			%	4	paragraph
			%	5	subparagraph



			\setlength{\parindent}{0cm} 		% 문서 들여 쓰기를 하지 않는다.


		%	--------------------------------------------------------------------------------------- 
		% 	mini toc 설정
		% 	
		% 	
		% 	--------------------------------------------------------- 장의 목차  minitoc package
			\usepackage{minitoc}

			\setcounter{minitocdepth}{4}    	%  Show until subsubsections in minitoc
%			\setcounter{minitocdepth}{5}    	%  Show until subsubsections in minitoc
%			\setlength{\mtcindent}{12pt} 	% default 24pt
			\setlength{\mtcindent}{24pt} 	% default 24pt

		% 	--------------------------------------------------------- part toc
		%	\setcounter{parttocdepth}{2} 	%  default
			\setcounter{parttocdepth}{0}
		%	\setlength{\ptcindent}{0em}		%  default  목차 내용 들여 쓰기
			\setlength{\ptcindent}{0em}         


		% 	--------------------------------------------------------- section toc

			\renewcommand{\ptcfont}{\normalsize\rm} 		%  default
			\renewcommand{\ptcCfont}{\normalsize\bf} 	%  default
			\renewcommand{\ptcSfont}{\normalsize\rm} 	%  default


		%	=======================================================================================
		% 	tocloft package
		% 	
		% 	------------------------------------------ 목차의 목차 번호와 목차 사이의 간격 조정
			\usepackage{tocloft}

		% 	------------------------------------------ 목차의 내어쓰기 즉 왼쪽 마진 설정
			\setlength{\cftsecindent}{2em}			%  section

		% 	------------------------------------------ 목차의 목차 번호와 목차 사이의 간격 조정
			\setlength{\cftsecnumwidth}{2em}		%  section





		%	=======================================================================================
		% 	flowchart  package
		% 	
		% 	------------------------------------------ 목차의 목차 번호와 목차 사이의 간격 조정
			\usepackage{flowchart}
			\usetikzlibrary{arrows}



		%	=======================================================================================
		% 	줄 간격 설정
		% 	
		% 	
		% 	--------------------------------- 	줄간격 설정
			\doublespace
%			\onehalfspace
%			\singlespace
		
		

	% 	============================================================================== itemi Global setting

	
		%	-------------------------------------------------------------------------------
		%		Vertical spacing
		%	-------------------------------------------------------------------------------
			\setlist[itemize]{topsep=0.0em}			% 상단의 여유치
			\setlist[itemize]{partopsep=0.0em}			% 
			\setlist[itemize]{parsep=0.0em}			% 
%			\setlist[itemize]{itemsep=0.0em}			% 
			\setlist[itemize]{noitemsep}				% 
			
		%	-------------------------------------------------------------------------------
		%		Horizontal spacing
		%	-------------------------------------------------------------------------------
			\setlist[itemize]{labelwidth=1em}			%  라벨의 표시 폭
			\setlist[itemize]{leftmargin=8em}			%  본문 까지의 왼쪽 여백  - 4em
			\setlist[itemize]{labelsep=3em} 			%  본문에서 라벨까지의 거리 -  3em
			\setlist[itemize]{rightmargin=0em}			% 오른쪽 여백  - 4em
			\setlist[itemize]{itemindent=0em} 			% 점 내민 거리 label sep 과 같은면 점위치 까지 내민다
			\setlist[itemize]{listparindent=3em}		% 본문 드려쓰기 간격
	
	
			\setlist[itemize]{ topsep=0.0em, 			%  상단의 여유치
						partopsep=0.0em, 		%  
						parsep=0.0em, 
						itemsep=0.0em, 
						labelwidth=1em, 
						leftmargin=2.5em,
						labelsep=2em,			%  본문에서 라벨 까지의 거리
						rightmargin=0em,		% 오른쪽 여백  - 4em
						itemindent=0em, 		% 점 내민 거리 label sep 과 같은면 점위치 까지 내민다
						listparindent=0em}		% 본문 드려쓰기 간격
	
%			\begin{itemize}
	
		%	-------------------------------------------------------------------------------
		%		Label
		%	-------------------------------------------------------------------------------
			\renewcommand{\labelitemi}{$\bullet$}
			\renewcommand{\labelitemii}{$\bullet$}
%			\renewcommand{\labelitemii}{$\cdot$}
			\renewcommand{\labelitemiii}{$\diamond$}
			\renewcommand{\labelitemiv}{$\ast$}		
	
%			\renewcommand{\labelitemi}{$\blacksquare$}   	% 사각형 - 찬것
%			\renewcommand\labelitemii{$\square$}		% 사각형 - 빈것	




			\usepackage{caption}
%			\captionsetup[table]{skip=10pt plus 0.01pt}
			\captionsetup[table]{skip=04pt plus 0.01pt}
			%\captionsetup{tableposition=above}
			






% ------------------------------------------------------------------------------
% Begin document (Content goes below)
% ------------------------------------------------------------------------------
	\begin{document}
	
			\dominitoc
			\doparttoc			




			\title{Pixlr}
			\author{김대희}
			\date{2019년 05월}
			\maketitle


			\tableofcontents 		% 목차 출력

			\cleardoublepage
			\listoffigures 			% 그림 목차 출력

			\cleardoublepage
			\listoftables 			% 표 목차 출력





		\mdfdefinestyle	{con_specification} {
						outerlinewidth		=1pt			,%
						innerlinewidth		=2pt			,%
						outerlinecolor		=blue!70!black	,%
						innerlinecolor		=white 			,%
						roundcorner			=4pt			,%
						skipabove			=1em 			,%
						skipbelow			=1em 			,%
						leftmargin			=0em			,%
						rightmargin			=0em			,%
						innertopmargin		=2em 			,%
						innerbottommargin 	=2em 			,%
						innerleftmargin		=1em 			,%
						innerrightmargin		=1em 			,%
						backgroundcolor		=gray!4			,%
						frametitlerule		=true 			,%
						frametitlerulecolor	=white			,%
						frametitlebackgroundcolor=black		,%
						frametitleaboveskip=1em 			,%
						frametitlebelowskip=1em 			,%
						frametitlefontcolor=white 			,%
						}





	% ===== ===== ===== ===== ===== ===== ===== ===== ===== ===== ===== ===== ===== ===== ===== ===== ===== ===== ===== ===== ===== ===== ===== ===== ===== ===== ===== ===== ===== ===== ===== ===== Part			pixlr
%		\addtocontents{toc}{\protect\newpage}
%		\part{Pixlr}
%		\noptcrule
%		\parttoc				




%	================================================================== chapter
%	\addtocontents{toc}{\protect\newpage}
	\chapter{버전의 종류} 
	\noptcrule
	\newpage
	\minitoc				


	% -----------------------------------------------------------------------------
	%
	% ----------------------------------------------------------------------------- 
		\section{버전의 종류}




\begin{itemize}[					
		topsep=0.0em,			
		parsep=0.0em,			
		itemsep=0em,			
		leftmargin=		6	em,
		labelwidth=3em,			
		labelsep=3em			
]					
	\item	https://pixlr.com/editor/
	\item	https://pixlr.com/x/
\end{itemize}					

	% -----------------------------------------------------------------------------
	%
	% ----------------------------------------------------------------------------- 
		\section{pixlr editor}

	% -----------------------------------------------------------------------------
	%
	% ----------------------------------------------------------------------------- 
		\section{pixlr x}






%	================================================================== chapter
%	\addtocontents{toc}{\protect\newpage}
	\chapter{	세팅	} 
	\noptcrule
	\newpage
	\minitoc				


	% -----------------------------------------------------------------------------
	%
	% ----------------------------------------------------------------------------- 
		\section{한글의 사용}


	% -----------------------------------------------------------------------------
	%
	% ----------------------------------------------------------------------------- 
		\section{	세팅				}


																		
% ------------------------------------------------------------------------------ table 																		
\begin{table} [h]																		
\caption{ pixlr : Settings }																		
\label{tab:title}																		
\tabulinesep= 1.2 em																		
% \tabulinesep= 0.0 em																		
\begin{tabu} to 1.0\linewidth {																		
	X [	r	,	1.00	]		% 01	내용										
	X [	l	,	1.00	]		% 02	단축키										
	X [	r	,	1.00	]		% 03	설명										
%	X [	r	,	1.00	]		% 04	비고										
%	X [	r	,	1.00	]		% 05	비고										
%	X[ r, 1.0 ] % 06 통장			% 06	지혜													
}																		
\hline \hline																		
			내용				&설명									&	비고	\\  \hline \hline
			Show Guide				&									&		\\  \hline
			Snap To Guide				&									&		\\  \hline
			Always Show Transform				&									&		\\  \hline
			Auto Select Layer				&									&		\\  \hline
			Light UI Mode				&									&		\\  \hline
			Keyboard Shortcuts				&									&		\\  \hline \hline 
\end{tabu}																		
\end{table}																		
% ===== ===== ===== ===== ===== ===== ===== ===== .																		
\clearpage																		
% ===== ===== ===== ===== ===== ===== ===== ===== .																		




%	================================================================== chapter 메뉴의구성
%	\addtocontents{toc}{\protect\newpage}
	\chapter{메뉴의 구성} 
	\noptcrule
	\newpage
	\minitoc				


	\chapter{	화면 구성	} 
	\minitoc
% -----------------------------------------------------------------------------												
\section{	화면 구성	}		


	\chapter{메뉴 : 파일} 
	\minitoc
% -----------------------------------------------------------------------------												
\section{	파일	}		

\section{	새 이미지		}									
\section{	이미지열기		}									
\section{	이미지 URL 열기		}									
\section{	저장		}									
\section{	인쇄		}									
\section{	종료		}									
\section{	로그인		}									
\section{	등록		}									
\section{	편집		}									


					
	\chapter{메뉴 : 편집} 
	\minitoc
% -----------------------------------------------------------------------------												
\section{	편집	}										
\section{	입력 취소		}									
\section{	다시 실행		}									
\section{	자르기		}									
\section{	복사		}									
\section{	삭제		}									
\section{	붙여넣기		}									
\section{	자유 변형		}									
\section{	자유 왜속		}									
\section{	모두 선택		}									
\section{	모두 선택 취소		}									
\section{	반전 선택		}									
\section{	select pixels		}									
\section{	브러쉬 정의		}									


	\chapter{메뉴 : 이미지} 
	\minitoc
% -----------------------------------------------------------------------------												
\section{	이미지	}										
\section{	이지지 크기				}							
\section{	캔버스 크기				}							
\section{	캔버스를 180도 회전				}							
\section{	캔버스를 시계방향으로 90도 회전				}							
\section{	캔버스를 시계반대방향으로 90도 회전				}							
\section{	캔버스를 수직으로 뒤집기				}							
\section{	캔버스를 수평으로 뒤집기				}							
\section{	오려내기				}							



	\chapter{메뉴 : 레이어} 
	\minitoc
% -----------------------------------------------------------------------------												
\section{	레이어	}										
\section{	새 레이어				}							
\section{	레이어 복제				}							
\section{	레이어 삭제				}							
\section{	이미지를 레이어로 열기				}							
\section{	이지지 URL을 레이어로 열기				}							
\section{	아래 레이어어와 병합				}							
\section{	보이는 레이어 병합				}							
\section{	배경으로 이미지 병합				}							
\section{	레이어를 위로 이동				}							
\section{	레이어를 아래로 이동				}							
\section{	레이어 스타일				}							
\section{	레이어 래스터화				}							
\section{	레이어 마스크 추가				}							
\section{	레이어 마스크 삭제				}							
\section{	레이어 마스크 적용				}							
\section{	레이어를 180도 회전				}							
\section{	레이어를 시계방향으로 90도 회전				}							
\section{	레이어를 시계반대방향으로 90도 회전				}							
\section{	레이어를 수직으로 뒤집기				}							
\section{	레이어를 수평으로 뒤집기				}							




	\chapter{메뉴 : 조정} 
	\minitoc
% -----------------------------------------------------------------------------												
	\section{	조정	}
																			
% ------------------------------------------------------------------------------ table 																			
\begin{table} [h]																			
\caption{ pixlr : 조정 }																			
\label{tab:title}																			
\tabulinesep= 0.5 em																			
% \tabulinesep= 0.0 em																			
\begin{tabu} to 1.0\linewidth {																			
	X [	r	,	1.00	]			% 01	내용										
	X [	r	,	1.00	]			% 02	단축키										
	X [	r	,	1.00	]			% 02	단축키										
	X [	r	,	1.00	]			% 03	설명										
}																			
\hline \hline																			
			구분		&			&	내용								&	비고	\\  \hline \hline
			 밝기 \& 명암대비 		&			&									&		\\  \hline
			 색상 \& 채도 		&			&									&		\\  \hline
			 Color balance 		&			&									&		\\  \hline
			 Color vibrance 		&			&									&		\\  \hline
			 레벨 		&			&									&		\\  \hline
			 커브 		&			&									&		\\  \hline
			 노출 		&			&									&		\\  \hline
\hline																			
			 자동 레벨 		&			&									&		\\  \hline
\hline																			
			 반전 		&			&									&		\\  \hline
			 세피아 		&			&									&		\\  \hline
			 과대노출 		&			&									&		\\  \hline
			 채도 제거 		&			&									&		\\  \hline
			 옛 사진 		&			&									&		\\  \hline
			 크로스 프로세스 		&			&									&		\\  \hline
			 임계치 		&			&									&		\\  \hline
			 포스터화 		&			&									&		\\  \hline
			 색 참조표 		&			&									&		\\  \hline
\hline																			
\end{tabu}																			
\end{table}																			
% ===== ===== ===== ===== ===== ===== ===== ===== .																			
\clearpage																			
% ===== ===== ===== ===== ===== ===== ===== ===== .																			
																			
																			
																			
																			
																			
																			
																			
																			
																			
																			



\section{	밝기 \& 명암대비				}							
\section{	색상 \& 채도				}							
\section{	Color balance				}							
\section{	Color vibrance				}							
\section{	레벨				}							
\section{	커브				}							
\section{	노출				}							
\section{	자동 레벨				}							
\section{	반전				}							
\section{	세피아				}							
\section{	과대노출				}							
\section{	채도 제거				}							
\section{	옛 사진				}							
\section{	크로스 프로세스				}							
\section{	임계치				}							
\section{	포스터화				}							
\section{	색 참조표				}							



										
	\chapter{메뉴 : 필터} 
	\minitoc
% -----------------------------------------------------------------------------												
	\section{	필터	}

																		

																			
% ------------------------------------------------------------------------------ table 																			
\begin{table} [h]																			
\caption{ pixlr :  필터 }																			
\label{tab:title}																			
\tabulinesep= 0.3 em																			
% \tabulinesep= 0.0 em																			
\begin{tabu} to 1.0\linewidth {																			
	X [	r	,	0.80	]			% 01	내용										
	X [	r	,	0.80	]			% 02	단축키										
	X [	r	,	1.80	]			% 02	단축키										
	X [	r	,	0.200	]			% 03	설명										
%	X [	r	,	1.00	]			% 04	비고										
%	X [	r	,	1.00	]			% 05	비고										
%	X[ r, 1.0 ] % 06 통장			% 06	지혜														
}																			
\hline \hline																			
			구분		&			&	내용								&	비고	\\  \hline \hline
			흐림		&	Blur		&									&		\\  \hline
			 사각형 흐림효과 		&	Box blur		&									&		\\  \hline
			 가우시안 흐림효과 		&	Gaussian blur		&									&		\\  \hline
\hline																			
			 선명 효과 		&	Sharpen		&선명하게 									&		\\  \hline
			 인샵마스크 		&	Unsharp mask		&사진의 색조간의 경계 부분을 뚜렷하게 해서 선명하게 만들어 주는 역할 &		\\  \hline
			 노이즈 제거 		&	Denoise		&									&		\\  \hline
			 노이즈 		&	Noise		&									&		\\  \hline
			 확산 		&	Diffuse		&광선 확산 필터 							&		\\  \hline
			 스캔라인 		&	Scan lines		&									&		\\  \hline
			 하프톤 		&	Half tone		&									&		\\  \hline
			 픽셀화 		&	Pixelate		&									&		\\  \hline
			 점묘화 		&	Pointnize		&									&		\\  \hline
			 소용돌이 		&	Water swirl		&									&		\\  \hline
			 극좌표 		&	Polar coordinates		&									&		\\  \hline
			 칼레이도스코프 		&	Kaleidoscope		&									&		\\  \hline
			 틸트 시프트 		&	Titt shift		&									&		\\  \hline
\hline																			
			 비네트 		&	Vignette		&									&		\\  \hline
			 파스텔 		&	Pastels		&									&		\\  \hline
			 글래머 글로우 		&	Galamiur glow		&									&		\\  \hline
			 HDR 흉내 		&	Mimic HDR		&									&		\\  \hline
			 희망 		&	Hope		&									&		\\  \hline
			 아트 포스터 		&	Art poster		&									&		\\  \hline
			 열지도 		&	Heat map		&									&		\\  \hline
			 트리톤 		&	Tritone		&									&		\\  \hline
			 Night vision 		&	Night vision		&									&		\\  \hline
\hline																			
			 영각 		&	Emboss		&									&		\\  \hline
			 인간 		&	Engrave		&									&		\\  \hline
			 모서리 찾기 		&	Find edges		&									&		\\  \hline
\hline																			
\end{tabu}																			
\end{table}																			
% ===== ===== ===== ===== ===== ===== ===== ===== .																			
\clearpage																			
% ===== ===== ===== ===== ===== ===== ===== ===== .																			
																			
																			

																	
																		
																		


\section{	흐림				}							
\section{	사각형 흐림효과				}							
\section{	가우시안 흐림효과				}							
\section{	선명 효과				}							
\section{	인샵마스크				}
사진을 보정할 때 마지막으로 선명한 결과물을 얻기 위해 활용, 사진의 색조간의 경계 부분을 뚜렷하게 해서 선명하게 만들어 주는 역할 
							
\section{	노이즈 제거				}							
\section{	노이즈 제거				}							
\section{	확산				}							
\section{	스캔라인				}							
\section{	하프톤				}
포토샵의 컬러하프톤 필터는 점차적으로 흐려지는 이미지와 사용을 하면 망점효과도 점점 작아지는 효과가 만들어집니다. 
흐려지는 이미지는 부드러운 브러시를 이용하거나 그래디언트, 가우시안 블러를 이용하면 가능하게 됩니다. 
또한 선택툴의 페더기능을 이용해도 같은 효과를 얻을 수 있습니다. 
이번 글에서는 여러가지 다양한 하프톤효과와 알파채널과 선택툴의 페더기능을 이용해서 글자에 하프톤효과를 만드는 방법을 알아봅니다. 
출처: https://martian36.com/725 [웹디자인 \& 포토샵]


\section{	픽셀화				}					
\section{	점묘화				}							
\section{	소용돌이				}							
\section{	극좌표				}							
\section{	칼레이도스코프				}							
\section{	틸트 시프트				}							
\section{	비네트				}							
\section{	파스텔				}							
\section{	글래머 글로우				}							
\section{	HDR 흉내				}							
\section{	희망				}							
\section{	아트 포스터				}							
\section{	열지도				}							
\section{	트리톤				}							
\section{	Night vision				}							
\section{	영각				}							
\section{	인간				}							
\section{	모서리 찾기				}							



	\chapter{메뉴 : 보기} 
	\minitoc
% -----------------------------------------------------------------------------												
\section{	보기	}
\section{	줌인				}							
\section{	줌아웃				}							
\section{	실제 픽셀				}							
\section{	모두 보기				}							
\section{	네비게이터				}							
\section{	레이어				}							
\section{	히스토리				}							
\section{	도구 옵션				}							
\section{	전체화면 모드				}							
\section{	팔레트 위치 재설정				}							



	\chapter{메뉴 : 언어} 
	\minitoc
% -----------------------------------------------------------------------------												
\section{	언어	}


	\chapter{메뉴 : 도움말} 
	\minitoc
% -----------------------------------------------------------------------------												
\section{	도움말	}
\section{	도움말				}							
\section{	문의처				}							



	\chapter{메뉴 : 글꼴} 
	\minitoc
% -----------------------------------------------------------------------------												
\section{	글꼴	}										
\section{	무료 글꼴				}							
\section{	프리미엄 글꼴				}							
\section{	글꼴 번들				}							
\section{	자신의 글꼴을 사용하는 방법				}							



	\chapter{메뉴 : 공짜} 
	\minitoc
% -----------------------------------------------------------------------------												
\section{	공짜	}										
\section{	Free 	Backgrounds			}							
\section{	Pixlr X-Photo Editor Made Easy				}							
\section{	LoveSVG - Free SVG Resources				}							
\section{	Vectr - Free Vector Graphics Software				}							
\section{	Free Images, Vectors, Footage \& Audio				}							



%	================================================================== chapter 편집
%	\addtocontents{toc}{\protect\newpage}
	\chapter{편집} 
	\noptcrule
	\newpage
	\minitoc				



	\chapter{단축키} 
	\minitoc

% -----------------------------------------------------------------------------												
	\section{	단축키				}
																		
% ------------------------------------------------------------------------------ table 																		
\begin{table} [h]																		
\caption{ pixlr : Keyboard Shortcuts }																		
\label{tab:title}																		
\tabulinesep= 0.4 em																		
% \tabulinesep= 0.0 em																		
\begin{tabu} to 1.0\linewidth {																		
	X [	r	,	1.40	]		% 01	내용										
	X [	l	,	1.00	]		% 02	단축키										
	X [	r	,	1.00	]		% 03	설명										
	X [	r	,	1.00	]		% 04	비고										
%	X [	r	,	1.00	]		% 05	비고										
%	X[ r, 1.0 ] % 06 통장			% 06	지혜													
}																		
\hline \hline																		
			내용				&	단축 키				&	설명			&	비고	\\  \hline \hline
			Save				&	Ctrl	+	S		&				&		\\  \hline
			Close				&	Ctrl	+	Q		&				&		\\  \hline
			Close Tool				&	ESC				&				&		\\  \hline
			Delete Layer				&	DEL				&				&		\\  \hline \hline
			Zoom 	In			&	Ctrl	+	+		&				&		\\  \hline
			Zoom 	Out			&	Ctrl	+	-		&				&		\\  \hline
			Zoom 	Fity			&	Ctrl	+	0		&				&		\\  \hline
			Zoom 	1X			&	Ctrl	+	1		&				&		\\  \hline
			Zoom 	Fill			&	Ctrl	+	2		&				&		\\  \hline
			Zoom 	3X			&	Ctrl	+	3		&				&		\\  \hline \hline
			Pan / Move				&	Spacebar				&				&		\\  \hline
			Arrange				&	V				&				&		\\  \hline
			Crop				&	C				&				&		\\  \hline
			Cutout				&	K				&				&		\\  \hline
			Adjust				&	A				&				&		\\  \hline
			Filter				&	F				&				&		\\  \hline
			Effect				&	ESC				&				&		\\  \hline
			Retouch				&	R				&				&		\\  \hline
			Draw				&	B				&				&		\\  \hline
			Text				&	T				&				&		\\  \hline
			Add	Element			&	O				&				&		\\  \hline
			Add	Image			&	I				&				&		\\  \hline \hline 

			Undo				&	Ctrl	+	Z		&				&		\\  \hline
			Redo				&	Ctrl	+	Y		&				&		\\  \hline \hline

			Move Layer		(position)		&					&				&		\\  \hline 
			Up				&$\uparrow$			&				&		\\  \hline
			Down			&$\downarrow$					&				&		\\  \hline
			Left				&$\leftarrow$					&				&		\\  \hline
			Right				&$\rightarrow$					&				&		\\  \hline \hline
\end{tabu}																		
\end{table}																		
% ===== ===== ===== ===== ===== ===== ===== ===== .																		
\clearpage																		
% ===== ===== ===== ===== ===== ===== ===== ===== .																		







	\chapter{이미지} 
	\minitoc
% -----------------------------------------------------------------------------												
\section{	이미지				}							
\section{	이미지 불러오기				}							
\section{	이미지 저장하기				}							
\section{	이미지 확대하기				}							
\section{	이미지 축소하기				}							


\section{	이미지 선택하기				}							
\section{	이미지 해제하기				}							
\section{	이미지 선택 툴				}							




	\section{	색수차 현상				}							
	\section{	비네팅 현상				}							
	\section{	이미지 왜곡				}							




% ------------------------------------------------------------------------------
% End document
% ------------------------------------------------------------------------------
\end{document}



% =================================================================================================== Part 혼화 재료

% ========================================================================================= chapter

%	-----------------------------------------------------------  section  

	%	------------------------------------------------------------------------------  table

			\begin{table} [h]
	
			\caption{잔 골재의 표준입도}  
			\label{tab:title} 
	
			\begin{center}
			\tabulinesep=0.4em
			\begin{tabu} to 0.8\linewidth { X[r] X[l] X[c]  }
			\tabucline [1pt,] {-}
			체의 호칭 치수 (mm)		& 체를 통화한 것의 질량 백분율(\%) \\
			\multicolumn{4}{c} {단위량} \\
			\tabucline [0.1pt,] {-}
			2.5	&100\\
			1.2	& 99 $\sim$ 100 \\
			0.6	& 60 $\sim$ 80 \\
			0.3	& 20 $\sim$ 50 \\
			0.15	&  5 $\sim$ 30 \\
			\tabucline [0.1pt,] {-}
			\end{tabu} 
			\end{center}
			\end{table}



	%	------------------------------------------------------------------------------  문제

		\clearpage
		\begin{small}	
		\begin{question}
		레디 믹스트 콘크리트에서 \textbf{회수수}를 혼합수로 사용할 경우 주의할 사항 중 틀린것은 ?
		\begin{enumerate}[label=\arabic*), topsep=0.0em, itemsep=-1.0em ]
			\item [①] 고강도 콘크리트의 경우 회수수를 사용하여서는 안된다. 
			\item [②] 슬러지수의 사용 시 \textbf{단위 슬러지 고형분}은 콘크리트 질량의 3\% 이하로 한다. 
			\item [③] 회수수의 품질 시험 항목은 4가지로 염소 이온량, 시멘트 응결 시간의 차, 모르타르 압축강도의 비, 단위 슬러지 고형분율 이다. 
			\item [④] 콘크리트를 배합할 때, 회수수 중에 함유된 슬러지 고형분은 물의 질량에는 포함되지 않는다. 
		\end{enumerate}
		\end{question}



		\begin{solution}
		해설
		\end{solution}
		\end{small}	
		\hrulefill


		%	----------------------------------------------------------------------------- 수식
			\begin{equation}
			\begin{aligned}
			T_2 = T_1 - 0.15 ( T_1 - T_0 ) \times t
			\end{aligned}
			\end{equation}


			\begin{description}[style=sameline, leftmargin=2cm, topsep=0.0em, itemsep=0.0em]
			\item[$T_0$] 		주의의 온도(${}^\circ C$)
			\item[$T_1$] 		비볐을때의 콘크리트 온도 (${}^\circ C$)
			\item[$t$] 		비빈후 부터 타설이 끝났을때 까지의 시간 (h)
			\end{description}


			\begin{itemize}	 	[
							topsep=0.0em, 
							parsep=0.0em, 
							itemsep=0em, 
							leftmargin=12.0em, 
							labelwidth=3em, 
							labelsep=3em
							] 
			\item [] 1
			\item [] 2	
			\item [] 3	
			\item [] 4	
			\end{itemize}




			\begin{itemize}[topsep=0.0em, parsep=0.0em, itemsep=0em, leftmargin=12.0em, labelwidth=3em, labelsep=3em] 
			\item [1.] 	따다 아사나
			\item [2.] 	우드르바 하스타 아사나
			\item [3.] 	웃따나 아사나
			\item [4.] 	아르다 웃따나 아사나
			\item [5.] 	차뚜랑가 단다 아사나
			\item [6.] 	우르드바 우카 스바나 아사나
			\item [7.] 	아도무카 스바나 아사나
			\item [8.] 	아르다 웃따나 아사나
			\item [9.] 	웃따나 아사나
			\item [10.] 	우드르바 하스타 아사나
			\item [11.] 	따다 아사나
			\end{itemize}


	\href{https://www.youtube.com/watch?v=SpqKCQZQBcc}{태양경배자세A}
	\href{https://www.youtube.com/watch?v=CL3czAIUDFY}{태양경배자세A}


													

