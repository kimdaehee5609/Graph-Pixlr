%	------------------------------------------------------------------------------
%
%
%		픽슬러
%
%		2022년
%		8월
%		15일
%		월요일
%		첫작성
%
%
%

%	\documentclass[25pt, a1paper]{tikzposter}
%	\documentclass[25pt, a0paper, landscape]{tikzposter}
%	\documentclass[25pt, a1paper ]{tikzposter}
	\documentclass[	20pt, 
							a1paper, 
							portrait, %
							margin=0mm, %
							innermargin=10mm,  		%
%							blockverticalspace=4mm, %
							colspace=5mm, 
							subcolspace=0mm
							]{tikzposter}


%	\documentclass[25pt, a1paper]{tikzposter}
%	\documentclass[25pt, a1paper]{tikzposter}
%	\documentclass[25pt, a1paper]{tikzposter}

% 	12pt  14pt 17pt  20pt  25pt
%
%	a0 a1 a2
%
%	landscape  portrait
%

	%% Tikzposter is highly customizable: please see
	%% https://bitbucket.org/surmann/tikzposter/downloads/styleguide.pdf

	%	========================================================== 	Package
		\usepackage{kotex}						% 한글 사용


%% Available themes: see also
%% https://bitbucket.org/surmann/tikzposter/downloads/themes.pdf
%	\usetheme{Default}
%	\usetheme{Rays}
%	\usetheme{Basic}
	\usetheme{Simple}
%	\usetheme{Envelope}
%	\usetheme{Wave}
%	\usetheme{Board}
%	\usetheme{Autumn}
%	\usetheme{Desert}

%% Further changes to the title etc is possible
%	\usetitlestyle{Default}			%
%	\usetitlestyle{Basic}				%
%	\usetitlestyle{Empty}				%
%	\usetitlestyle{Filled}				%
%	\usetitlestyle{Envelope}			%
%	\usetitlestyle{Wave}				%
%	\usetitlestyle{verticalShading}	%


%	\usebackgroundstyle{Default}
%	\usebackgroundstyle{Rays}
%	\usebackgroundstyle{VerticalGradation}
%	\usebackgroundstyle{BottomVerticalGradation}
%	\usebackgroundstyle{Empty}

%	\useblockstyle{Default}
%	\useblockstyle{Basic}
%	\useblockstyle{Minimal}		% 이것은 간단함
%	\useblockstyle{Envelope}		% 
%	\useblockstyle{Corner}		% 사각형
%	\useblockstyle{Slide}			%	띠모양  
	\useblockstyle{TornOut}		% 손그림모양


	\usenotestyle{Default}
%	\usenotestyle{Corner}
%	\usenotestyle{VerticalShading}
%	\usenotestyle{Sticky}

%	\usepackage{fontspec}
%	\setmainfont{FreeSerif}
%	\setsansfont{FreeSans}

%	------------------------------------------------------------------------------ 제목

	\title{픽슬러}

	\author{ 	작성 : 2022.08.15.월 \\
			수정 : 2022.08.15.월 }

%	\institute{서영엔지니어링}
%	\titlegraphic{\includegraphics[width=7cm]{IMG_1934}}

	%% Optional title graphic
	%\titlegraphic{\includegraphics[width=7cm]{IMG_1934}}
	%% Uncomment to switch off tikzposter footer
	% \tikzposterlatexaffectionproofoff

\begin{document}

	\maketitle

	\begin{columns}

		\column{0.5}

%	------------------------------------------------------------------------------ 픽슬러
			\block{■  픽슬러  }
			{
					\setlength{\leftmargini}{7em}
					\setlength{\labelsep} {1em}
				\begin{LARGE}
					\begin{itemize}
					\item 
					\item 
					\item 
					\end{itemize}
				\end{LARGE}
			}

%	------------------------------------------------------------------------------ 
			\block{■    }
			{
					\setlength{\leftmargini}{7em}
					\setlength{\labelsep} {1em}
				\begin{LARGE}
					\begin{itemize}
					\item 
					\item 
					\item 
					\end{itemize}
				\end{LARGE}
			}





	%	====== ====== ====== ====== ====== 
		\column{0.5}


%	------------------------------------------------------------------------------ 
			\block{■    }
			{
					\setlength{\leftmargini}{7em}
					\setlength{\labelsep} {1em}
				\begin{LARGE}
					\begin{itemize}
					\item 
					\item 
					\item 
					\end{itemize}
				\end{LARGE}
			}




%	------------------------------------------------------------------------------ 파일
			\block{■  파일 }
			{
				\begin{LARGE}
					\begin{itemize}
					\item 폴더명 : P-Pixlr
					\item 파일명 : 픽슬러-포스트.tex
					\end{itemize}
				\end{LARGE}
			}


	\end{columns}




\end{document}


		\begin{huge}
		\end{huge}

		\begin{LARGE}
		\end{LARGE}

		\begin{Large}
		\end{Large}

		\begin{large}
		\end{large}

